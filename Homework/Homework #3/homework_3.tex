
\documentclass[12pt]{article}
 
\usepackage[margin=1in]{geometry}
\usepackage{amsmath,amsthm,amssymb, mathtools}
\usepackage[T1]{fontenc}
\usepackage{lmodern}
\usepackage{fixltx2e}
\usepackage[shortlabels]{enumitem}
\usepackage{mathrsfs}
\usepackage{kbordermatrix}

\usepackage{graphicx}
\usepackage{bbm}

\renewcommand{\kbldelim}{(}% Left delimiter
\renewcommand{\kbrdelim}{)}% Right delimiter
 
\newcommand{\N}{\mathbb{N}}
\newcommand{\R}{\mathbb{R}}
\newcommand{\Z}{\mathbb{Z}}
\newcommand{\Q}{\mathbb{Q}}
 
\newenvironment{theorem}[2][Theorem]{\begin{trivlist}
\item[\hskip \labelsep {\bfseries #1}\hskip \labelsep {\bfseries #2.}]}{\end{trivlist}}
\newenvironment{lemma}[2][Lemma]{\begin{trivlist}
\item[\hskip \labelsep {\bfseries #1}\hskip \labelsep {\bfseries #2.}]}{\end{trivlist}}
\newenvironment{exercise}[2][Exercise]{\begin{trivlist}
\item[\hskip \labelsep {\bfseries #1}\hskip \labelsep {\bfseries #2.}]}{\end{trivlist}}
\newenvironment{problem}[2][Problem]{\begin{trivlist}
\item[\hskip \labelsep {\bfseries #1}\hskip \labelsep {\bfseries #2.}]}{\end{trivlist}}
\newenvironment{question}[2][Question]{\begin{trivlist}
\item[\hskip \labelsep {\bfseries #1}\hskip \labelsep {\bfseries #2.}]}{\end{trivlist}}
\newenvironment{corollary}[2][Corollary]{\begin{trivlist}
\item[\hskip \labelsep {\bfseries #1}\hskip \labelsep {\bfseries #2.}]}{\end{trivlist}}
\newcommand{\textfrac}[2]{\dfrac{\text{#1}}{\text{#2}}}
\newcommand{\floor}[1]{\left\lfloor #1 \right\rfloor}

\newenvironment{amatrix}[1]{%
  \left(\begin{array}{@{}*{#1}{c}|c@{}}
}{%
  \end{array}\right)
}

\DeclareMathOperator*{\E}{\mathbb{E}}


\begin{document}

\title{Dynamical Systems II: Homework 3}

\author{Chris Hayduk}
\date{February 25, 2021}

\maketitle

\section{Questions from Silva}

\subsection{Section 2.5}

\begin{problem}{14}
\end{problem}

We have that $E = \{\emptyset\} \subset X$. We also have that $\{\emptyset\} \in \mathcal{S}$. Hence, with $A, B = \{\emptyset\}$ we trivially have that $$A \subset E \subset B$$ and $$\mu(B \setminus A) = 0$$

Thus, $E = \{\emptyset\} \in \mathcal{S}_{\mu}$ and $\mathcal{S}_{\mu}$ is non-empty. Now fix $E \in \mathcal{S}_{\mu}$ and consider $E^C$. Since $E \in S_{\mu}$, we have $A, B \in \mathcal{S}$ such that $$A \subset E \subset B$$ and $$\mu(B \setminus A) = 0$$

Applying the complement to the above expression gives us that $A^C \supset E^C \supset B^C$. So we want to show that $\mu(A^C \setminus B^C) = 0$. But note that $A^C \setminus B^C = A^C \cap B$ and $B \setminus A = B \cap A^C$, so we have that,
\begin{align*}
\mu(A^C \setminus B^C) &= \mu(A^C \cap B)\\
&= \mu(B \setminus A)\\
&= 0
\end{align*}

Hence, $\mathcal{S}_{\mu}$ is closed under complements.\\

Now let $E_n \in \mathcal{S}_{\mu}$, $n \geq 1$ and consider $\cup_{n=1}^{\infty} E_n$. For each $n$, we have that there exists $A_n, B_n \in S$ such that $A_n \subset E_n \subset B_n$ and $\mu(B_n \setminus A_n) = 0$. Since $E_n \subset B_n$ for all $n$, we must have that $\cup E_n \subset \cup B_n$. Similarly, we have $\cup A_n \subset \cup E_n$.\\

Now we need to show that $\mu(\cup B_n \setminus \cup A_n) = 0$. By Corollary 2.4.2 and countable subadditivity, we have,
\begin{align*}
\mu(\cup B_n \setminus \cup A_n) &= \mu(\cup B_n) - \mu(\cup A_n)\\
&\leq \mu(B_1) + \mu(B_2) + \cdots - \mu(A_1) - \mu(A_2) - \mu(A_3) - \cdots\\
&= \mu(B_1) - \mu(A_1) + \mu(B_2) - \mu(A_2) + \mu(B_3) - \mu(A_3) + \cdots\\
&= \mu(B_1 \setminus A_1) + \mu(B_2 \setminus A_2) + \mu(B_3 \setminus A_3) + \cdots\\
&= 0 + 0 + 0 + \cdots\\
&= 0
\end{align*}

Hence, $\cup E_n \in \mathcal{S}_{\mu}$ and so $\mathcal{S}_{\mu}$ is closed under countable unions. Hence, $\mathcal{S}_{\mu}$ is a $\sigma$-algebra. Now fix $E \in \mathcal{S}$. Then if we set $A, B = E \in \mathcal{S}$, we have that $A \subset E \subset B$ and $\mu(B \setminus A) = \mu(E \setminus E) = \mu(\emptyset) = 0$ as required. Thus, every element in $\mathcal{S}$ is also in $\mathcal{S}_{\mu}$, and so $\mathcal{S}_{\mu}$ is a $\sigma$-algebra containing $\mathcal{S}$.

\begin{problem}{15}
\end{problem}

Define $\overline{\mu}$ on elements of $S_{\mu}$ by $\overline{\mu}(E) = \mu(A)$ for any $A \in \mathcal{S}$ such that there is a $B \in \mathcal{S}$ with $A \subset E \subset B$ and $\mu(B \setminus A) = 0$.\\

Fix $E \in S_{\mu}$ and let $A, B \in \mathcal{S}$ be any two sets such that $A \subset E \subset B$ and $\mu(B \setminus A) = 0$.\\

Since $E \subset B$, we have that $E \setminus A \subset B \setminus A$. Thus,
\begin{align*}
\mu(E \setminus A) \leq \mu(B \setminus A) = 0
\end{align*}

For any other choice of $A', B' \in \mathcal{S}$ such that $A' \subset E \subset B'$ and $\mu(B' \setminus A') = 0$, we also have that $\mu(E \setminus A') = 0$. Since $A, A' \subset E$, we have,
\begin{align*}
\mu(A' \triangle A) &\leq \mu(E \setminus A)\\
&= 0
\end{align*}

Hence, we have that $\mu(A' \setminus A) = 0 = \mu(A \setminus A')$. Now note that,
\begin{align*}
\mu(A') &= \mu(A \setminus (A \setminus A')\\
&= \mu(A) - \mu(A \setminus A')\\
&= \mu(A) - 0\\
&= \mu(A)
\end{align*}

Thus, we have that $\overline{\mu}(E)$ is independent of the choice of $A$ and $B$.\\

Now fix $E_1 \in \mathcal{S}_{\mu}$ with $\mu(E_1) = 0$ and consider $E_2 \subset E_1$.

\subsection{Section 2.6}

\begin{problem}{2}
\end{problem}

Note that the rational numbers are countable and that for each $q \in \mathbb{Q}$, we have that $\{q\}$ is closed. Thus, $\{q\}^C$ is open for all $q \in \mathbb{Q}$. Hence, if we take,
\begin{align*}
\bigcap_{q \in \mathbb{Q}} \{q\}^C
\end{align*}

we see that we are taking a countable intersection of open sets. This is precisely the definition of a $G_{\delta}$ set. Moreover, this set is the intersection of the complements of the rational numbers. That is, the only numbers in $\mathbb{R}$ not included in this set are the rationals. Hence, we have constructed the irrational numbers.\\

Note the irrational numbers are not open. If we fix an irrational number $x$, observe that for any value of $\epsilon > 0$, we will find a rational number $q$ such that $q \in B(x, \epsilon)$. This follows directly from the fact that $\mathbb{Q}$ is dense in $\mathbb{R}$. Moreover, the irrational numbers are not closed. The irrationals are also dense in $\mathbb{R}$, so by definition we can find a sequence of irrational numbers converging to any real number. If we pick $q \in \mathbb{Q} \subset \mathbb{R}$, then we see that we can construct a sequence of irrational numbers which converge to $q \not\in \bigcap_{q \in \mathbb{Q}} \{q\}^C$. Thus, the irrational numbers are neither open nor closed as required.

\begin{problem}{5}
\end{problem}

Let $\mathcal{C}$ be the set of all closed intervals with rational endpoints. That is, intervals of the form $\left[ \frac{p_1}{q_1}, \frac{p_2}{q_2} \right]$ with $p_1, q_1, p_2, q_2 \in \mathbb{Z}$.

\subsection{Section 2.7}

\begin{problem}{1}
\end{problem}

Suppose that $\mathcal{C}$ is a semi-ring of subsets of a nonempty set $X$ and $\emptyset \neq Y \subset X$. Consider the collection $\{A \cap Y \ : \ A \in \mathcal{C} \}$ and suppose that this collection is non-empty.\\

Suppose $C, D \in \{A \cap Y : A \in \mathcal{C} \}$. We have $C = A \cap Y$ and $D = B \cap Y$ for some $A, B \in \mathcal{C}$. Consider the following,
\begin{align*}
C \cap D &= (A \cap Y) \cap (B \cap Y)\\
&= (A \cap B) \cap Y
\end{align*}

Since $\mathcal{C}$ is a semi-ring, we have that $A \cap B \in \mathcal{C}$ and so $C \cap D = (A \cap B) \cap Y$ is in our collection.\\

Now consider $C \setminus D$. We can express this as the following,
\begin{align*}
C \setminus D &= (A \cap Y) \setminus (B \cap Y)\\
&= (A \setminus B) \cap Y
\end{align*}

Since $\mathcal{C}$ is a semi-ring, we have that $A \setminus B$ can be expressed as $$\sqcup_{j=1}^n E_j$$ where $E_j \in \mathcal{C}$ for every $j$. Thus, we can rewrite the previous statement as,
\begin{align*}
C \setminus D &= (A \setminus B) \cap Y\\
&= (\sqcup_{j=1}^n E_j) \cap Y\\
&= \sqcup_{j=1}^n (E_j \cap Y)
\end{align*}

Since $E_j \in \mathcal{C}$, we have that $E_j \cap Y \in \{A \cap Y \ : \ A \in \mathcal{C} \}$. Furthermore, since $E_j \in \mathcal{C}$ are disjoint, $E_j \cap Y \in \{A \cap Y \ : \ A \in \mathcal{C} \}$ are disjoint as well. Therefore, $C \setminus D$ satisfies all the required properties for a semi-ring.\\

Now let us extend to the case where $\mathcal{C}$ is a ring. We now need to show that our new collection is closed under finite unions. It suffices to show that the collection is closed for a single union (i.e. $C \cup D$) because we can then extend this union a finite number of times $n$. So now let us consider $C \cup D$,
\begin{align*}
C \cup D &= (A \cap Y) \cup (B \cap Y)\\
&= (A \cup B) \cap Y
\end{align*}

Since $\mathcal{C}$ is a ring, we have that $A \cup B \in \mathcal{C}$ and so $(A \cup B) \cap Y$ is in our collection. Hence, $\{A \cap Y \ : \ A \in \mathcal{C} \}$ is a ring.

\begin{problem}{2}
\end{problem}

Let $(X, \mathcal{L}, \lambda)$ be a canonical Lebesgue measure space and $\mathcal{C}$ a sufficient semi-ring. Now for any nonempty measurable set $X_0 \subset X$, consider $\mathcal{C} \cap X_0 = \{C \cap X_0 : C \in \mathcal{C}\}$. We want to show that this set is a sufficient semi-ring for $(X_0, \mathcal{L}(X_0), \lambda)$.\\

Fix $Y \in \mathcal{C} \cap X_0$. Then $Y = C \cap X_0$ for some $C \in \mathcal{C}$ and so $y \in Y \iff y \in X_0$ and $y \in C$. Thus, $Y \subset X_0$ by definition and, as a result, every set in $\mathcal{C} \cap X_0$ is a subset of $X_0$.\\

Now since $\mathcal{C}$ is a semi-ring, we have $\emptyset \in \mathcal{C}$. Hence, $\emptyset \cap X_0 = \emptyset \in \mathcal{C} \cap X_0$, and so the collection is non-empty.\\

Fix $A, B \in \mathcal{C} \cap X_0$. Then $A = C_1 \cap X_0$, $B = C_2 \cap X_0$ for some $C_1, C_2 \in \mathcal{C}$. We have,
\begin{align*}
A \cap B &= (C_1 \cap X_0) \cap (C_2 \cap X_0)\\
&= (C_1 \cap C_2) \cap X_0
\end{align*}

Since $\mathcal{C}$ is a sufficient semi-ring, we have that $C_1 \cap C_2 \in \mathcal{C}$ and thus $(C_1 \cap C_2) \cap X_0 \in \mathcal{C} \cap X_0$.\\

Now consider $A \setminus B$. Note that this is given by $(C_1 \cap X_0) \setminus (C_2 \cap X_0)$. This is equivalent to $(C_1 \setminus C_2) \cap X_0$. Since $C_1, C_2 \in \mathcal{C}$, we have,
\begin{align*}
C_1 \setminus C_2 &= \sqcup_{j=1}^n E_j
\end{align*}

where $E_j \in \mathcal{C}$ are disjoint. Rewriting our formulation for $A \setminus B$ yields,
\begin{align*}
A \setminus B &= (C_1 \setminus C_2) \cap X_0\\
&= (\sqcup_{j=1}^n E_j) \cap X_0\\
&= \sqcup_{j=1}^n (E_j \cap X_0)
\end{align*}

Note that $E_j \cap X_0 \in \mathcal{C} \cap X_0$ for every $j$. Also note that the sets $E_j \cap X_0$ are disjoint. To prove this, fix $i$ and $j$ and suppose $x \in E_j \cap X_0$ and $x \in E_i \cap X_0$. Then by the definition of intersection, we have that $x \in E_j$, $x \in E_i$, and $x \in X_0$. But we know that $E_i, E_j$ are disjoint, so $x$ cannot be in both sets. Thus we have a contradiction and the sets $E_j \cap X_0$ are disjoint as required.\\

Hence we have shown that $\mathcal{C} \cap X_0$ is a semi-ring. Now we need to show that it satisfies the sufficient semi-ring property. First we need to show that every set contained in $\mathcal{C} \cap X_0$ has finite measure. Consider $A$ as defined above. We have,
\begin{align*}
\lambda(A) &= \lambda(C_1 \cap X_0)\\
&\leq \lambda(C_1) < \infty
\end{align*}

Hence, $A$ has finite measure. Recall that $\mathcal{C}$ is a sufficient semi-ring. Then $$\lambda(C_1) = \inf \left\{ \sum_{j=1}^{\infty} \lambda(I_j) \ : \ C_1 \subset \cup_{j=1}^{\infty}I_j \text{ and } I_j \in \mathcal{C} \text{ for } j \geq 1 \right\}$$ Applying this to the definition of $A$ yields,
\begin{align*}
\lambda(A) &= \lambda(C_1 \cap X_0)\\
&= \inf \left\{ \sum_{j=1}^{\infty} \lambda(I_j \cap X_0) \ : \ A \subset \cup_{j=1}^{\infty} I_j \cap X_0 \text{ and } I_j \cap X_0 \in \mathcal{C} \cap X_0 \text{ for } j \geq 1 \right\}
\end{align*} 

as required.

\begin{problem}{6}
\end{problem}

\begin{problem}{7}
\end{problem}

\subsection{Section 3.3}

\begin{problem}{1}
\end{problem}

Define $T: [0, 1] \to [0,1]$ by $T(x) = 2x$ if $0 \leq x \leq 1/2$ and $T(x) = 2 - 2x$ if $1/2 < x < 1$. Define $S_1: [0, 1] \to [0, 1/2]$ by $S_1(y) = y/2$ and $S_2: [0, 1] \to [1/2, 1]$ by $S_2(y) = y/2 + 1/2$. For a measurable set $A \subset [0, 1]$. We have that,
\begin{align*}
T^{-1}(A) &= S_1(A) \sqcup (S_2(A \setminus \{0\}))
\end{align*}

Since we are taking a singleton away from $A$ and $S_2$ is a well-defined function, this will not change the measure of $S_2(A)$. Thus, from the above and from the results in Chapter 2, we have,
\begin{align*}
\lambda(T^{-1}(A)) &= \lambda(S_1(A)) + \lambda(S_2(A \setminus \{0\}))\\
&= \lambda(\frac{1}{2}A) + \lambda(\frac{1}{2}A + \frac{1}{2})\\
&= \frac{1}{2}\lambda(A) + \frac{1}{2}\lambda(A)\\
&= \lambda(A)
\end{align*}

Thus, $T$ is measure preserving.



\end{document}