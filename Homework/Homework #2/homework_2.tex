\documentclass[12pt]{article}
 
\usepackage[margin=1in]{geometry}
\usepackage{amsmath,amsthm,amssymb, mathtools}
\usepackage[T1]{fontenc}
\usepackage{lmodern}
\usepackage{fixltx2e}
\usepackage[shortlabels]{enumitem}
\usepackage{mathrsfs}
\usepackage{kbordermatrix}

\usepackage{graphicx}
\usepackage{bbm}

\renewcommand{\kbldelim}{(}% Left delimiter
\renewcommand{\kbrdelim}{)}% Right delimiter
 
\newcommand{\N}{\mathbb{N}}
\newcommand{\R}{\mathbb{R}}
\newcommand{\Z}{\mathbb{Z}}
\newcommand{\Q}{\mathbb{Q}}
 
\newenvironment{theorem}[2][Theorem]{\begin{trivlist}
\item[\hskip \labelsep {\bfseries #1}\hskip \labelsep {\bfseries #2.}]}{\end{trivlist}}
\newenvironment{lemma}[2][Lemma]{\begin{trivlist}
\item[\hskip \labelsep {\bfseries #1}\hskip \labelsep {\bfseries #2.}]}{\end{trivlist}}
\newenvironment{exercise}[2][Exercise]{\begin{trivlist}
\item[\hskip \labelsep {\bfseries #1}\hskip \labelsep {\bfseries #2.}]}{\end{trivlist}}
\newenvironment{problem}[2][Problem]{\begin{trivlist}
\item[\hskip \labelsep {\bfseries #1}\hskip \labelsep {\bfseries #2.}]}{\end{trivlist}}
\newenvironment{question}[2][Question]{\begin{trivlist}
\item[\hskip \labelsep {\bfseries #1}\hskip \labelsep {\bfseries #2.}]}{\end{trivlist}}
\newenvironment{corollary}[2][Corollary]{\begin{trivlist}
\item[\hskip \labelsep {\bfseries #1}\hskip \labelsep {\bfseries #2.}]}{\end{trivlist}}
\newcommand{\textfrac}[2]{\dfrac{\text{#1}}{\text{#2}}}
\newcommand{\floor}[1]{\left\lfloor #1 \right\rfloor}

\newenvironment{amatrix}[1]{%
  \left(\begin{array}{@{}*{#1}{c}|c@{}}
}{%
  \end{array}\right)
}

\DeclareMathOperator*{\E}{\mathbb{E}}


\begin{document}

\title{Dynamical Systems II: Homework 2}

\author{Chris Hayduk}
\date{February 18, 2021}

\maketitle

\section{Questions from Silva}

\subsection{Section 2.3}

\begin{problem}{1}
\end{problem}

Suppose $A$ is measurable. Then for any $\epsilon > )$, there is an open set $G = G_{\epsilon}$ such that $A \subset G$ and $\lambda^*(G \setminus A) < \epsilon$.

\begin{problem}{2}
\end{problem}

Suppose $A$ is a null set. Then for any $\epsilon > 0$, there exists a sequence of intervals $I_j$ such that $A \subset \bigcup_{j=1}^{\infty} I_j$ and $\sum_{j=1}^{\infty} | I_j | < \epsilon$. Now consider $A^2 = \{a^2 : \ a \in A\}$. Note that if $a \in I_k = (x, y)$ for some $k$, then
\begin{align*}
&x < a < y
\end{align*}

\begin{problem}{3}
\end{problem}

Suppose $A$ is any set, $B$ is measurable, and $\lambda^*(A \triangle B) = 0$. Note that $A \triangle B = (A \setminus B) \cup (B \setminus A)$. Also note that $(A \setminus B) \cup (B \setminus A) = \emptyset$. So we have,
\begin{align*}
\lambda^*(A \triangle B) &= \lambda*(() \cup (B \setminus A))\\
&= \lambda^*(A \setminus B) + \lambda^*(B \setminus A)\\
&= 0
\end{align*}

Thus, we have that $\lambda^*(A \setminus B) = -\lambda^*(B \setminus A)$. Since outer measure is non-negative, we must have that, $$\lambda^*(A \setminus B) = 0 = \lambda^*(B \setminus A)$$

So we have that $A \setminus B$ is a null set and is thus measurable. Now note that $A = (A \cap B) \cup (A \setminus B)$ and that $(A \cap B) \cap (A \setminus B) = \emptyset$. Since $B$ is measurable, we have that there exists an open set $G$ such that $B \subset G$ and $\lambda^*(G \setminus B) < \epsilon$

\begin{problem}{6}
\end{problem}

\begin{problem}{9}
\end{problem}

\subsection{Section 2.4}

\begin{problem}{2}
\end{problem}

\begin{problem}{4}
\end{problem}

\subsection{Section 2.5}

\begin{problem}{1}
\end{problem}

\begin{problem}{6}
\end{problem}

\begin{problem}{7}
\end{problem}

\begin{problem}{10}
\end{problem}

\begin{problem}{13}
\end{problem}

\end{document}