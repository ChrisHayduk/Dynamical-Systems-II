\documentclass[12pt]{article}
 
\usepackage[margin=1in]{geometry}
\usepackage{amsmath,amsthm,amssymb, mathtools}
\usepackage[T1]{fontenc}
\usepackage{lmodern}
\usepackage{fixltx2e}
\usepackage[shortlabels]{enumitem}
\usepackage{mathrsfs}
\usepackage{kbordermatrix}

\usepackage{graphicx}
\usepackage{bbm}

\renewcommand{\kbldelim}{(}% Left delimiter
\renewcommand{\kbrdelim}{)}% Right delimiter
 
\newcommand{\N}{\mathbb{N}}
\newcommand{\R}{\mathbb{R}}
\newcommand{\Z}{\mathbb{Z}}
\newcommand{\Q}{\mathbb{Q}}
 
\newenvironment{theorem}[2][Theorem]{\begin{trivlist}
\item[\hskip \labelsep {\bfseries #1}\hskip \labelsep {\bfseries #2.}]}{\end{trivlist}}
\newenvironment{lemma}[2][Lemma]{\begin{trivlist}
\item[\hskip \labelsep {\bfseries #1}\hskip \labelsep {\bfseries #2.}]}{\end{trivlist}}
\newenvironment{exercise}[2][Exercise]{\begin{trivlist}
\item[\hskip \labelsep {\bfseries #1}\hskip \labelsep {\bfseries #2.}]}{\end{trivlist}}
\newenvironment{problem}[2][Problem]{\begin{trivlist}
\item[\hskip \labelsep {\bfseries #1}\hskip \labelsep {\bfseries #2.}]}{\end{trivlist}}
\newenvironment{question}[2][Question]{\begin{trivlist}
\item[\hskip \labelsep {\bfseries #1}\hskip \labelsep {\bfseries #2.}]}{\end{trivlist}}
\newenvironment{corollary}[2][Corollary]{\begin{trivlist}
\item[\hskip \labelsep {\bfseries #1}\hskip \labelsep {\bfseries #2.}]}{\end{trivlist}}
\newcommand{\textfrac}[2]{\dfrac{\text{#1}}{\text{#2}}}
\newcommand{\floor}[1]{\left\lfloor #1 \right\rfloor}

\newenvironment{amatrix}[1]{%
  \left(\begin{array}{@{}*{#1}{c}|c@{}}
}{%
  \end{array}\right)
}

\newcommand{\Mod}[1]{\ (\mathrm{mod}\ #1)}


\DeclareMathOperator*{\E}{\mathbb{E}}


\begin{document}

\title{Dynamical Systems II: Homework 11}

\author{Chris Hayduk}
\date{May 4, 2021}

\maketitle

\section{Questions from Silva}

\subsection{Section 4.5}

\begin{problem}{2}
\end{problem}

\begin{problem}{3}
\end{problem}

\subsection{Section 5.1}

\begin{problem}{2}
\end{problem}

Let $T: X \to X$ be a measure-preserving transformation with $\mu(X) = 1$ and suppose that for every measurable set $A$ the limit
\begin{align*}
\lim_{n \to \infty} \frac{1}{n} \sum_{i=0}^{n-1} \mathbb{I}_A(T^i(x))
\end{align*}

exists and equals $\mu(A)$ a.e. We want to show that $T$ is ergodic. Let us fix two measurable sets $A, B$. Other than on two sets of measure $0$ (whose union is measure $0$), we have,
\begin{align*}
\mu(A) \mu(B) &= \lim_{n \to \infty} \frac{1}{n} \sum_{i=0}^{n-1} \mathbb{I}_A(T^i(x)) \cdot \lim_{n \to \infty} \frac{1}{n} \sum_{i=0}^{n-1} \mathbb{I}_B(T^i(x))\\
&=  \lim_{n \to \infty} \frac{1}{n} \sum_{i=0}^{n-1} \sum_{j=0}^{n-1}  \mathbb{I}_A(T^i(x)) \mathbb{I}_B(T^j(x))\\
&= \lim_{n \to \infty} \frac{1}{n} \sum_{i=0}^{n-1} \mathbb{I}_{A \cap B}(T^i(x))\\
&= 
\end{align*}

\begin{problem}{3}
\end{problem}

\subsection{Section 5.2}

\begin{problem}{2}
\end{problem}

\begin{problem}{4}
\end{problem}

\begin{problem}{5}
\end{problem}

\end{document}