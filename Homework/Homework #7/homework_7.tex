\documentclass[12pt]{article}
 
\usepackage[margin=1in]{geometry}
\usepackage{amsmath,amsthm,amssymb, mathtools}
\usepackage[T1]{fontenc}
\usepackage{lmodern}
\usepackage{fixltx2e}
\usepackage[shortlabels]{enumitem}
\usepackage{mathrsfs}
\usepackage{kbordermatrix}

\usepackage{graphicx}
\usepackage{bbm}

\renewcommand{\kbldelim}{(}% Left delimiter
\renewcommand{\kbrdelim}{)}% Right delimiter
 
\newcommand{\N}{\mathbb{N}}
\newcommand{\R}{\mathbb{R}}
\newcommand{\Z}{\mathbb{Z}}
\newcommand{\Q}{\mathbb{Q}}
 
\newenvironment{theorem}[2][Theorem]{\begin{trivlist}
\item[\hskip \labelsep {\bfseries #1}\hskip \labelsep {\bfseries #2.}]}{\end{trivlist}}
\newenvironment{lemma}[2][Lemma]{\begin{trivlist}
\item[\hskip \labelsep {\bfseries #1}\hskip \labelsep {\bfseries #2.}]}{\end{trivlist}}
\newenvironment{exercise}[2][Exercise]{\begin{trivlist}
\item[\hskip \labelsep {\bfseries #1}\hskip \labelsep {\bfseries #2.}]}{\end{trivlist}}
\newenvironment{problem}[2][Problem]{\begin{trivlist}
\item[\hskip \labelsep {\bfseries #1}\hskip \labelsep {\bfseries #2.}]}{\end{trivlist}}
\newenvironment{question}[2][Question]{\begin{trivlist}
\item[\hskip \labelsep {\bfseries #1}\hskip \labelsep {\bfseries #2.}]}{\end{trivlist}}
\newenvironment{corollary}[2][Corollary]{\begin{trivlist}
\item[\hskip \labelsep {\bfseries #1}\hskip \labelsep {\bfseries #2.}]}{\end{trivlist}}
\newcommand{\textfrac}[2]{\dfrac{\text{#1}}{\text{#2}}}
\newcommand{\floor}[1]{\left\lfloor #1 \right\rfloor}

\newenvironment{amatrix}[1]{%
  \left(\begin{array}{@{}*{#1}{c}|c@{}}
}{%
  \end{array}\right)
}

\newcommand{\Mod}[1]{\ (\mathrm{mod}\ #1)}


\DeclareMathOperator*{\E}{\mathbb{E}}


\begin{document}

\title{Dynamical Systems II: Homework 7}

\author{Chris Hayduk}
\date{April 13, 2021}

\maketitle

\section{Questions from Silva}

\subsection{Section 3.7}

\begin{problem}{2}
\end{problem}

Observe that the only $T$-invariant subset of $X$ is $X$ itself. Suppose we have a subset $A$ of $X$ such that $A \neq X$. Then there exists $a_j \in X$ such that $a_{(j+1) \text{ mod } n} \not\in A$ and hence, $T(a_j) \not\in A$. Thus $A$ is not $T$-invariant. Hence, we must have $A = X$. But $A^C = X^C = \emptyset$. So $\mu(A^C) = 0$ whenever $A$ is $T$-invariant, and so $T$ is ergodic.

\begin{problem}{3}
\end{problem}

Suppose $T$ is a totally ergodic measure-preserving transformation and suppose that $T$ is invertible. Since $T$ is ergodic, observe that $\mu(A) = 0$ or $\mu(A^C) = 0$ for any $T$-invariant set $A$. Now fix a $T$-invariant set $A$ and suppose $\mu(A) = 0$. Since $T$ is measure-preserving, we have that if $\mu(A) = 0$, then $\mu(T^{-1}(A)) = 0$.\\

For $T^{-2}(A)$, observe that we have $$T^{-2}(A) = T^{-1}(T^{-1}(A)) = T^{-1}(B)$$ where $B = T^{-1}(A)$, a measure $0$ set. Again, since $T$ is measure-preserving and $\mu(B) = 0$, we have that $\mu(T^{-1}(B)) = \mu(T^{-2}(A)) = 0$. Proceeding inductively, we get that if $\mu(A) = 0$, then $\mu(T^{n}(A)) = 0$ for all $n < 0$. Moreover, if $A$ is $T$-invariant, then $A$ is also $T^{-1}$-invariant, since $T$ is invertible. This again applies for $T^n$ with any $n < 0$, so we have that

\begin{problem}{6}
\end{problem}

Recall the the two dimensional Baker's transformation is defined as 
\begin{align*}
T(x, y) = \begin{cases}
(2x, \frac{y}{2}), \ \text{ if } 0 \leq x < 1/2\\
(2x-1, \frac{y+1}{2}), \ \text{ if } 1/2 \leq x \leq 1
\end{cases}
\end{align*}

For any subset $A$ of $[0,1] \times [0,1]$, we have that $\mu(A) = $

\subsection{Section 3.10}

\begin{problem}{1}
\end{problem}

\begin{problem}{3}
\end{problem}

\begin{problem}{4}
\end{problem}

\begin{problem}{6}
\end{problem}


\end{document}