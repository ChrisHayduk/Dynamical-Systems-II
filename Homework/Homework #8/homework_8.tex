\documentclass[12pt]{article}
 
\usepackage[margin=1in]{geometry}
\usepackage{amsmath,amsthm,amssymb, mathtools}
\usepackage[T1]{fontenc}
\usepackage{lmodern}
\usepackage{fixltx2e}
\usepackage[shortlabels]{enumitem}
\usepackage{mathrsfs}
\usepackage{kbordermatrix}

\usepackage{graphicx}
\usepackage{bbm}

\renewcommand{\kbldelim}{(}% Left delimiter
\renewcommand{\kbrdelim}{)}% Right delimiter
 
\newcommand{\N}{\mathbb{N}}
\newcommand{\R}{\mathbb{R}}
\newcommand{\Z}{\mathbb{Z}}
\newcommand{\Q}{\mathbb{Q}}
 
\newenvironment{theorem}[2][Theorem]{\begin{trivlist}
\item[\hskip \labelsep {\bfseries #1}\hskip \labelsep {\bfseries #2.}]}{\end{trivlist}}
\newenvironment{lemma}[2][Lemma]{\begin{trivlist}
\item[\hskip \labelsep {\bfseries #1}\hskip \labelsep {\bfseries #2.}]}{\end{trivlist}}
\newenvironment{exercise}[2][Exercise]{\begin{trivlist}
\item[\hskip \labelsep {\bfseries #1}\hskip \labelsep {\bfseries #2.}]}{\end{trivlist}}
\newenvironment{problem}[2][Problem]{\begin{trivlist}
\item[\hskip \labelsep {\bfseries #1}\hskip \labelsep {\bfseries #2.}]}{\end{trivlist}}
\newenvironment{question}[2][Question]{\begin{trivlist}
\item[\hskip \labelsep {\bfseries #1}\hskip \labelsep {\bfseries #2.}]}{\end{trivlist}}
\newenvironment{corollary}[2][Corollary]{\begin{trivlist}
\item[\hskip \labelsep {\bfseries #1}\hskip \labelsep {\bfseries #2.}]}{\end{trivlist}}
\newcommand{\textfrac}[2]{\dfrac{\text{#1}}{\text{#2}}}
\newcommand{\floor}[1]{\left\lfloor #1 \right\rfloor}

\newenvironment{amatrix}[1]{%
  \left(\begin{array}{@{}*{#1}{c}|c@{}}
}{%
  \end{array}\right)
}

\newcommand{\Mod}[1]{\ (\mathrm{mod}\ #1)}


\DeclareMathOperator*{\E}{\mathbb{E}}


\begin{document}

\title{Dynamical Systems II: Homework 8}

\author{Chris Hayduk}
\date{April 22, 2021}

\maketitle

\section{Questions from Silva}

\subsection{Section 3.11}

\begin{problem}{3}
Revised problem: Let $(X, \mathcal{S}, \mu, T)$ be an invertible, recurrent, finite measure-preserving dynamical system. If $A$ is a set of positive measure such that the transformation $T_A$ is ergodic on $A$ and $\mu(X \setminus \cup_{n \geq 0} T^{-n}(A)) = 0$, then $T$ is ergodic.\\

Note: $X \setminus \cup_{n \geq 0} T^{-n}(A)$ is the set of points that never hit $A$.
\end{problem}


\subsection{Section 4.2}

\begin{problem}{3}
\end{problem}

Suppose $A$ is measurable and consider $\mathbb{I}_A$. Observe that $\mathbb{I}_A(x) = 1$ if $x \in A$ and $\mathbb{I}_A(x) = 0$ if $x \not\in A$. Then we must have that,
\begin{align*}
\{x \in X \ : \ \mathbb{I}_A(x) > 0\} = A
\end{align*}

Hence, this set must be measurable and thus, by Proposition 4.2.1, we have that $\mathbb{I}_A$ is measurable.\\

Now suppose $\mathbb{I}_A$ is measurable. Then again by Proposition 4.2.1, we can say that $\{x \in X \ : \ \mathbb{I}_A(x) > 0\}$ is measurable. Since this set is equal to $A$ by the definition of $\mathbb{I}_A$, we must have that $A$ is measurable as well.

\begin{problem}{5}
\end{problem}

Note that $f^{-1}(A \cup B) = \{x \in X: f(x) \in A \cup B\}$. But note that $f(x) \in A \cup B$ is the same as $f(x) \in A$ or $f(x) \in B$ (inclusive or). Hence we have that,
\begin{align*}
f^{-1}(A \cup B) &= \{x \in X: f(x) \in A \cup B\}\\
&= \{x \in X: f(x) \in A \text{ or } f(x) \in B\}\\
&= \{x \in X: f(x) \in A\} \cup \{x \in X: f(x) \in B\}\\
&= f^{-1}(A) \cup f^{-1}(B)
\end{align*}

Now consider $f^{-1}(A \cap B)$. We have that,
\begin{align*}
f^{-1}(A \cap B) &= \{x \in X: f(x) \in A \cap B\}\\
&= \{x \in X: f(x) \in A \text{ and } f(x) \in B\}\\
&= \{x \in X: f(x) \in A\} \cap \{x \in X: f(x) \in B\}\\
&= f^{-1}(A) \cap f^{-1}(B)
\end{align*}

Lastly, consider $f^{-1}(\mathbb{R} \setminus A)$. We have that,
\begin{align*}
f^{-1}(\mathbb{R} \setminus A) &= \{x \in X: f(x) \in \mathbb{R} \setminus A\}\\
&= \{x \in X: f(x) \in A^c\}
\end{align*}

Note that the final line in the aboe is the set of $x$ in $X$ that map to $A^c$ in $\mathbb{R}$. In other words, it is the complement of the set of points that map to $A$ in $\mathbb{R}$. Hence, we have,
\begin{align*}
\{x \in X: f(x) \in A^c\} &= \{x \in X: f(x) \in A\}^c\\
&= f^{-1}(A)^c\\
&= X \setminus f^{-1}(A)
\end{align*}

as required.\\

Now consider $f(A \cup B)$. Let $x \in f(A \cup B)$. Then $x \in \{y \in \mathbb{R}: f^{-1}(y) \in A \cup B)$. Thus, $x \in A$ or $x \in B$, or both. That is, $x \in \{y \in \mathbb{R}: f^{-1}(y) \in A)$ or $x \in \{y \in \mathbb{R}: f^{-1}(y) \in A)$ or both. Hence, we have that $f(A \cup B) \subset f(A) \cup f(B)$. Now fix $x \in f(A) \cup f(B)$. Then $f^{-1}(x) \in A$ or $f^{-1}(x) \in B$. But this is exactly the definition of $f(A \cup B)$ and so $x \in f(A \cup B)$. Thus, $f(A) \cup B \subset f(A \cup B)$ and hence, $f(A \cup B) = f(A) \cup f(B)$.\\

For $f(A \cap B)$, let us use a counterexample. Consider $f(x) = x^2$ and $A = [-1, 0]$, $B = [0,1]$. Then $f(A \cap B) = f(\{0\}) = 0$, but $f(A) \cap f(B) = [0, 1] \cap [0,1] = [0, 1]$. Hence, these are not equal and this does not hold.\\

Lastly, consider $f(X \setminus A)$. Again let us use $f(x) = x^2$ as a counterexample with $X = \mathbb{R}$ and $A = (0,1]$. Then $f(X \setminus A) = \mathbb{R}$. However, $f(X) \setminus A = \mathbb{R} \setminus [0, 1]$. Hence, these are not equal and so this does not hold.

\begin{problem}{6}
\end{problem}

Suppose that $f$ is Lebesgue measurable. Then, by Proposition 4.2.1 and Lemma 4.2.2, the inverse image under $f$ of any interval is a measurable set. Now fix $G \in \mathbb{R}$ such that $G$ is an open set. Since every open subset of $\mathbb{R}$ is a countable union of disjoint open intervals, we have that $G = \sqcup_{k=1}^{\infty} I_k$ for some disjoint open intervals $I_k$. Now note that disjoint set in a function's image must have disjoint preimages. Otherwise, there would be an $x$ such that $f(x)$ has two outputs, which is not possible for a validly defined function. Hence, we must have that,
\begin{align*}
f^{-1}(G) &= f^{-1}\left(\sqcup_{k=1}^{\infty} I_k \right)\\
&= \sqcup_{k=1}^{\infty} f^{-1}(I_k)
\end{align*}

Since each $I_k$ is an interval, we have that $f^{-1}(I_k)$ is measurable. And since the countable union of measurable sets is measurable, we have that $f^{-1}(I_k) = f^{-1}(G)$ is measurable, as required.\\

Now suppose $f^{-1}(G)$ is measurable for every open set $G \subset \mathbb{R}$. Then, in particular, the preimage of every open interval in $\mathbb{R}$ is measurable. Hence, $$\{x \in X \ : \ f(x) < a\}$$ is measurable for all $a \in \mathbb{R}$ and so $f$ is measurable.

\begin{problem}{7}
\end{problem}

Suppose $g: \mathbb{R} \to \mathbb{R}$ is Lebesgue measurable and $f: \mathbb{R} \to \mathbb{R}$ is continuous. We want to show that $f \circ g = f(g(x))$ is Lebesgue measurable.\\

Note that $\mathbb{R}$ is a metric space and $f$ is continuous on $\mathbb{R}$. Hence, by Lemma 4.2.3, $f$ is Lebesgue measurable as well. Now fix $B \in \mathcal{B}(\mathbb{R})$. We have that,
\begin{align*}
(f \circ g)^{-1}(B) &= g^{-1}(f^{-1}(B))
\end{align*} 

We have that $f^{-1}(B) \in \mathcal{L}(\mathbb{R})$ since $f$ is Lebesgue measurable. We need to use the continuity of $f$ to show that $f^{-1}(B)$ is Borel.\\

A Borel set is any set that can be formed from open sets through the operations of countable union, countable intersection, and complement. The inverse images of open sets under a continuous function are open sets and inverse images of a countable union is the countable union of the inverse images. The same notions hold true for complements and countable intersections. Hence, we can write $f^{-1}(B)$ as an expression of open sets through countable unions, countable intersections, and complements.\\

Hence, $f^{-1}(B)$ is a Borel set and so $g^{-1}(f^{-1}(B))$ is Lebesgue measurable, and so we have that $f \circ g$ is measurable, as required.

\begin{problem}{8}
\end{problem}

Suppose $f, g: \mathbb{R} \to \mathbb{R}$ are Lebesgue measurable and $g$ is such that for all null sets $N$, $g^{-1}(N)$ is measurable. We want to show that $f \circ g = f(g(x))$ is Lebesgue measurable.\\

Fix $B \in \mathcal{B}(\mathbb{R})$. We have that,
\begin{align*}
(f \circ g)^{-1}(B) &= g^{-1}(f^{-1}(B))
\end{align*} 

We have that $f^{-1}(B) \in \mathcal{L}(\mathbb{R})$ since $f$ is Lebesgue measurable. We need to use the property of $g$ to show that $g^{-1}(f^{-1}(B))$ is measurable.\\

Since $f^{-1}(B)$ is measurable, then there exists a $G_{\delta}$ set $G^*$ and a null set $N$ such that $f^{-1}(B) = G^* \setminus N = G^* \cap N^C$. Note that $G^*$ is a countable intersection of open sets, and hence is Borel. Hence, we have from Problem 5 that,
\begin{align*}
g^{-1}(f^{-1}(B)) &= g^{-1}(G^* \cap N^C)\\
&= g^{-1}(G^*) \cap g^{-1}(N)^C
\end{align*}

Since $G^*$ is a Borel set, we have that $g^{-1}(G^*)$ is measurable and since $N$ is a null set, we have that $g^{-1}(N)^C$ is measurable. Thus, we have that $g^{-1}(f^{-1}(B))$ is a finite intersection of measurable sets and hence is measurable. As a result, $f \circ g$ is Lebesgue measurable.

\begin{problem}{9}
\end{problem}

Suppose that $f$ is a Lebesgue measurable function. Then by Proposition 4.2.1, we have that $$\{x \in X \ : \ f(x) \geq a\}$$ and $$\{x \in X \ : \ f(x) \leq a\}$$ are both measurable sets for any $a \in \mathbb{R}$. Since the Lebesgue measurable sets form a sigma algebra, we can take the intersection of these sets and still have a measurable set. This gives us that,
\begin{align*}
\{x \in X \ : \ f(x) \geq a\} \cap \{x \in X \ : \ f(x) \leq a\} &= \{x \in X \ : \ f(x) = a\}
\end{align*}

is measurable for every $a \in \mathbb{R}$, as required.

\begin{problem}{10}
\end{problem}

Let $x \in \{x \in X \ : \ \lim_{n \to \infty} f_n(x) > \alpha\}$ Then $\lim_{n \to \infty} f_n(x)$ converges to some number $f(x)$ such that $f(x) > \alpha$. Hence, if we fix $\epsilon > 0$, there exists $N \in \mathbb{N}$ such that for all $n > N$, we have,
\begin{align*}
|f_n(x) - f(x)| < |f_n(x) - \alpha| < \epsilon
\end{align*}

\end{document}